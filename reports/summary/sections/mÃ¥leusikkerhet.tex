\begin{formal}
Vektmålingene står særdeles sentralt i analysene og følgelig er det viktig å ha en kvantifisert forståelse av måleusikkerheten knyttet til vekt. Dette kan for eksempel gjøres ved å gå gjennom et 10-talls turer og se på maksimal, og minimal vekt for en og samme kjøretur med samme lass. Differansen mellom minimal og maksimal måling gir et uttrykk for måleusikkerheten. Og ved å samestille data fra 10+ turer få får man en forståelse av måleusikkerheten. Kommer man opp i 30+  turer begynner man å få nok data til å kunne si at resultatene er statistisk valide.
\end{formal}
Ettersom det ikke ligger til grunn noen \enquote{kalibreringsturer} der man med sikkerhet vet at kjøretøyet har
kjørt med den samme vekten hele turen, er dette forløpig ikke sett på.
Kjøretøyene har dog logget \enquote{posisjonsdata}, altså logging av ulike attributter (deriblant vekt) mot posisjon.
Denne dataen kan trolig benyttes til dette formålet, og kan være et gjøremål når nye ressurser er i gang i slutten av mai.
